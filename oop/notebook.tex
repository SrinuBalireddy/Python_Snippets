
% Default to the notebook output style

    


% Inherit from the specified cell style.




    
\documentclass[11pt]{article}

    
    
    \usepackage[T1]{fontenc}
    % Nicer default font (+ math font) than Computer Modern for most use cases
    \usepackage{mathpazo}

    % Basic figure setup, for now with no caption control since it's done
    % automatically by Pandoc (which extracts ![](path) syntax from Markdown).
    \usepackage{graphicx}
    % We will generate all images so they have a width \maxwidth. This means
    % that they will get their normal width if they fit onto the page, but
    % are scaled down if they would overflow the margins.
    \makeatletter
    \def\maxwidth{\ifdim\Gin@nat@width>\linewidth\linewidth
    \else\Gin@nat@width\fi}
    \makeatother
    \let\Oldincludegraphics\includegraphics
    % Set max figure width to be 80% of text width, for now hardcoded.
    \renewcommand{\includegraphics}[1]{\Oldincludegraphics[width=.8\maxwidth]{#1}}
    % Ensure that by default, figures have no caption (until we provide a
    % proper Figure object with a Caption API and a way to capture that
    % in the conversion process - todo).
    \usepackage{caption}
    \DeclareCaptionLabelFormat{nolabel}{}
    \captionsetup{labelformat=nolabel}

    \usepackage{adjustbox} % Used to constrain images to a maximum size 
    \usepackage{xcolor} % Allow colors to be defined
    \usepackage{enumerate} % Needed for markdown enumerations to work
    \usepackage{geometry} % Used to adjust the document margins
    \usepackage{amsmath} % Equations
    \usepackage{amssymb} % Equations
    \usepackage{textcomp} % defines textquotesingle
    % Hack from http://tex.stackexchange.com/a/47451/13684:
    \AtBeginDocument{%
        \def\PYZsq{\textquotesingle}% Upright quotes in Pygmentized code
    }
    \usepackage{upquote} % Upright quotes for verbatim code
    \usepackage{eurosym} % defines \euro
    \usepackage[mathletters]{ucs} % Extended unicode (utf-8) support
    \usepackage[utf8x]{inputenc} % Allow utf-8 characters in the tex document
    \usepackage{fancyvrb} % verbatim replacement that allows latex
    \usepackage{grffile} % extends the file name processing of package graphics 
                         % to support a larger range 
    % The hyperref package gives us a pdf with properly built
    % internal navigation ('pdf bookmarks' for the table of contents,
    % internal cross-reference links, web links for URLs, etc.)
    \usepackage{hyperref}
    \usepackage{longtable} % longtable support required by pandoc >1.10
    \usepackage{booktabs}  % table support for pandoc > 1.12.2
    \usepackage[inline]{enumitem} % IRkernel/repr support (it uses the enumerate* environment)
    \usepackage[normalem]{ulem} % ulem is needed to support strikethroughs (\sout)
                                % normalem makes italics be italics, not underlines
    

    
    
    % Colors for the hyperref package
    \definecolor{urlcolor}{rgb}{0,.145,.698}
    \definecolor{linkcolor}{rgb}{.71,0.21,0.01}
    \definecolor{citecolor}{rgb}{.12,.54,.11}

    % ANSI colors
    \definecolor{ansi-black}{HTML}{3E424D}
    \definecolor{ansi-black-intense}{HTML}{282C36}
    \definecolor{ansi-red}{HTML}{E75C58}
    \definecolor{ansi-red-intense}{HTML}{B22B31}
    \definecolor{ansi-green}{HTML}{00A250}
    \definecolor{ansi-green-intense}{HTML}{007427}
    \definecolor{ansi-yellow}{HTML}{DDB62B}
    \definecolor{ansi-yellow-intense}{HTML}{B27D12}
    \definecolor{ansi-blue}{HTML}{208FFB}
    \definecolor{ansi-blue-intense}{HTML}{0065CA}
    \definecolor{ansi-magenta}{HTML}{D160C4}
    \definecolor{ansi-magenta-intense}{HTML}{A03196}
    \definecolor{ansi-cyan}{HTML}{60C6C8}
    \definecolor{ansi-cyan-intense}{HTML}{258F8F}
    \definecolor{ansi-white}{HTML}{C5C1B4}
    \definecolor{ansi-white-intense}{HTML}{A1A6B2}

    % commands and environments needed by pandoc snippets
    % extracted from the output of `pandoc -s`
    \providecommand{\tightlist}{%
      \setlength{\itemsep}{0pt}\setlength{\parskip}{0pt}}
    \DefineVerbatimEnvironment{Highlighting}{Verbatim}{commandchars=\\\{\}}
    % Add ',fontsize=\small' for more characters per line
    \newenvironment{Shaded}{}{}
    \newcommand{\KeywordTok}[1]{\textcolor[rgb]{0.00,0.44,0.13}{\textbf{{#1}}}}
    \newcommand{\DataTypeTok}[1]{\textcolor[rgb]{0.56,0.13,0.00}{{#1}}}
    \newcommand{\DecValTok}[1]{\textcolor[rgb]{0.25,0.63,0.44}{{#1}}}
    \newcommand{\BaseNTok}[1]{\textcolor[rgb]{0.25,0.63,0.44}{{#1}}}
    \newcommand{\FloatTok}[1]{\textcolor[rgb]{0.25,0.63,0.44}{{#1}}}
    \newcommand{\CharTok}[1]{\textcolor[rgb]{0.25,0.44,0.63}{{#1}}}
    \newcommand{\StringTok}[1]{\textcolor[rgb]{0.25,0.44,0.63}{{#1}}}
    \newcommand{\CommentTok}[1]{\textcolor[rgb]{0.38,0.63,0.69}{\textit{{#1}}}}
    \newcommand{\OtherTok}[1]{\textcolor[rgb]{0.00,0.44,0.13}{{#1}}}
    \newcommand{\AlertTok}[1]{\textcolor[rgb]{1.00,0.00,0.00}{\textbf{{#1}}}}
    \newcommand{\FunctionTok}[1]{\textcolor[rgb]{0.02,0.16,0.49}{{#1}}}
    \newcommand{\RegionMarkerTok}[1]{{#1}}
    \newcommand{\ErrorTok}[1]{\textcolor[rgb]{1.00,0.00,0.00}{\textbf{{#1}}}}
    \newcommand{\NormalTok}[1]{{#1}}
    
    % Additional commands for more recent versions of Pandoc
    \newcommand{\ConstantTok}[1]{\textcolor[rgb]{0.53,0.00,0.00}{{#1}}}
    \newcommand{\SpecialCharTok}[1]{\textcolor[rgb]{0.25,0.44,0.63}{{#1}}}
    \newcommand{\VerbatimStringTok}[1]{\textcolor[rgb]{0.25,0.44,0.63}{{#1}}}
    \newcommand{\SpecialStringTok}[1]{\textcolor[rgb]{0.73,0.40,0.53}{{#1}}}
    \newcommand{\ImportTok}[1]{{#1}}
    \newcommand{\DocumentationTok}[1]{\textcolor[rgb]{0.73,0.13,0.13}{\textit{{#1}}}}
    \newcommand{\AnnotationTok}[1]{\textcolor[rgb]{0.38,0.63,0.69}{\textbf{\textit{{#1}}}}}
    \newcommand{\CommentVarTok}[1]{\textcolor[rgb]{0.38,0.63,0.69}{\textbf{\textit{{#1}}}}}
    \newcommand{\VariableTok}[1]{\textcolor[rgb]{0.10,0.09,0.49}{{#1}}}
    \newcommand{\ControlFlowTok}[1]{\textcolor[rgb]{0.00,0.44,0.13}{\textbf{{#1}}}}
    \newcommand{\OperatorTok}[1]{\textcolor[rgb]{0.40,0.40,0.40}{{#1}}}
    \newcommand{\BuiltInTok}[1]{{#1}}
    \newcommand{\ExtensionTok}[1]{{#1}}
    \newcommand{\PreprocessorTok}[1]{\textcolor[rgb]{0.74,0.48,0.00}{{#1}}}
    \newcommand{\AttributeTok}[1]{\textcolor[rgb]{0.49,0.56,0.16}{{#1}}}
    \newcommand{\InformationTok}[1]{\textcolor[rgb]{0.38,0.63,0.69}{\textbf{\textit{{#1}}}}}
    \newcommand{\WarningTok}[1]{\textcolor[rgb]{0.38,0.63,0.69}{\textbf{\textit{{#1}}}}}
    
    
    % Define a nice break command that doesn't care if a line doesn't already
    % exist.
    \def\br{\hspace*{\fill} \\* }
    % Math Jax compatability definitions
    \def\gt{>}
    \def\lt{<}
    % Document parameters
    \title{Classes \& Instances  - 1 and 2}
    
    
    

    % Pygments definitions
    
\makeatletter
\def\PY@reset{\let\PY@it=\relax \let\PY@bf=\relax%
    \let\PY@ul=\relax \let\PY@tc=\relax%
    \let\PY@bc=\relax \let\PY@ff=\relax}
\def\PY@tok#1{\csname PY@tok@#1\endcsname}
\def\PY@toks#1+{\ifx\relax#1\empty\else%
    \PY@tok{#1}\expandafter\PY@toks\fi}
\def\PY@do#1{\PY@bc{\PY@tc{\PY@ul{%
    \PY@it{\PY@bf{\PY@ff{#1}}}}}}}
\def\PY#1#2{\PY@reset\PY@toks#1+\relax+\PY@do{#2}}

\expandafter\def\csname PY@tok@w\endcsname{\def\PY@tc##1{\textcolor[rgb]{0.73,0.73,0.73}{##1}}}
\expandafter\def\csname PY@tok@c\endcsname{\let\PY@it=\textit\def\PY@tc##1{\textcolor[rgb]{0.25,0.50,0.50}{##1}}}
\expandafter\def\csname PY@tok@cp\endcsname{\def\PY@tc##1{\textcolor[rgb]{0.74,0.48,0.00}{##1}}}
\expandafter\def\csname PY@tok@k\endcsname{\let\PY@bf=\textbf\def\PY@tc##1{\textcolor[rgb]{0.00,0.50,0.00}{##1}}}
\expandafter\def\csname PY@tok@kp\endcsname{\def\PY@tc##1{\textcolor[rgb]{0.00,0.50,0.00}{##1}}}
\expandafter\def\csname PY@tok@kt\endcsname{\def\PY@tc##1{\textcolor[rgb]{0.69,0.00,0.25}{##1}}}
\expandafter\def\csname PY@tok@o\endcsname{\def\PY@tc##1{\textcolor[rgb]{0.40,0.40,0.40}{##1}}}
\expandafter\def\csname PY@tok@ow\endcsname{\let\PY@bf=\textbf\def\PY@tc##1{\textcolor[rgb]{0.67,0.13,1.00}{##1}}}
\expandafter\def\csname PY@tok@nb\endcsname{\def\PY@tc##1{\textcolor[rgb]{0.00,0.50,0.00}{##1}}}
\expandafter\def\csname PY@tok@nf\endcsname{\def\PY@tc##1{\textcolor[rgb]{0.00,0.00,1.00}{##1}}}
\expandafter\def\csname PY@tok@nc\endcsname{\let\PY@bf=\textbf\def\PY@tc##1{\textcolor[rgb]{0.00,0.00,1.00}{##1}}}
\expandafter\def\csname PY@tok@nn\endcsname{\let\PY@bf=\textbf\def\PY@tc##1{\textcolor[rgb]{0.00,0.00,1.00}{##1}}}
\expandafter\def\csname PY@tok@ne\endcsname{\let\PY@bf=\textbf\def\PY@tc##1{\textcolor[rgb]{0.82,0.25,0.23}{##1}}}
\expandafter\def\csname PY@tok@nv\endcsname{\def\PY@tc##1{\textcolor[rgb]{0.10,0.09,0.49}{##1}}}
\expandafter\def\csname PY@tok@no\endcsname{\def\PY@tc##1{\textcolor[rgb]{0.53,0.00,0.00}{##1}}}
\expandafter\def\csname PY@tok@nl\endcsname{\def\PY@tc##1{\textcolor[rgb]{0.63,0.63,0.00}{##1}}}
\expandafter\def\csname PY@tok@ni\endcsname{\let\PY@bf=\textbf\def\PY@tc##1{\textcolor[rgb]{0.60,0.60,0.60}{##1}}}
\expandafter\def\csname PY@tok@na\endcsname{\def\PY@tc##1{\textcolor[rgb]{0.49,0.56,0.16}{##1}}}
\expandafter\def\csname PY@tok@nt\endcsname{\let\PY@bf=\textbf\def\PY@tc##1{\textcolor[rgb]{0.00,0.50,0.00}{##1}}}
\expandafter\def\csname PY@tok@nd\endcsname{\def\PY@tc##1{\textcolor[rgb]{0.67,0.13,1.00}{##1}}}
\expandafter\def\csname PY@tok@s\endcsname{\def\PY@tc##1{\textcolor[rgb]{0.73,0.13,0.13}{##1}}}
\expandafter\def\csname PY@tok@sd\endcsname{\let\PY@it=\textit\def\PY@tc##1{\textcolor[rgb]{0.73,0.13,0.13}{##1}}}
\expandafter\def\csname PY@tok@si\endcsname{\let\PY@bf=\textbf\def\PY@tc##1{\textcolor[rgb]{0.73,0.40,0.53}{##1}}}
\expandafter\def\csname PY@tok@se\endcsname{\let\PY@bf=\textbf\def\PY@tc##1{\textcolor[rgb]{0.73,0.40,0.13}{##1}}}
\expandafter\def\csname PY@tok@sr\endcsname{\def\PY@tc##1{\textcolor[rgb]{0.73,0.40,0.53}{##1}}}
\expandafter\def\csname PY@tok@ss\endcsname{\def\PY@tc##1{\textcolor[rgb]{0.10,0.09,0.49}{##1}}}
\expandafter\def\csname PY@tok@sx\endcsname{\def\PY@tc##1{\textcolor[rgb]{0.00,0.50,0.00}{##1}}}
\expandafter\def\csname PY@tok@m\endcsname{\def\PY@tc##1{\textcolor[rgb]{0.40,0.40,0.40}{##1}}}
\expandafter\def\csname PY@tok@gh\endcsname{\let\PY@bf=\textbf\def\PY@tc##1{\textcolor[rgb]{0.00,0.00,0.50}{##1}}}
\expandafter\def\csname PY@tok@gu\endcsname{\let\PY@bf=\textbf\def\PY@tc##1{\textcolor[rgb]{0.50,0.00,0.50}{##1}}}
\expandafter\def\csname PY@tok@gd\endcsname{\def\PY@tc##1{\textcolor[rgb]{0.63,0.00,0.00}{##1}}}
\expandafter\def\csname PY@tok@gi\endcsname{\def\PY@tc##1{\textcolor[rgb]{0.00,0.63,0.00}{##1}}}
\expandafter\def\csname PY@tok@gr\endcsname{\def\PY@tc##1{\textcolor[rgb]{1.00,0.00,0.00}{##1}}}
\expandafter\def\csname PY@tok@ge\endcsname{\let\PY@it=\textit}
\expandafter\def\csname PY@tok@gs\endcsname{\let\PY@bf=\textbf}
\expandafter\def\csname PY@tok@gp\endcsname{\let\PY@bf=\textbf\def\PY@tc##1{\textcolor[rgb]{0.00,0.00,0.50}{##1}}}
\expandafter\def\csname PY@tok@go\endcsname{\def\PY@tc##1{\textcolor[rgb]{0.53,0.53,0.53}{##1}}}
\expandafter\def\csname PY@tok@gt\endcsname{\def\PY@tc##1{\textcolor[rgb]{0.00,0.27,0.87}{##1}}}
\expandafter\def\csname PY@tok@err\endcsname{\def\PY@bc##1{\setlength{\fboxsep}{0pt}\fcolorbox[rgb]{1.00,0.00,0.00}{1,1,1}{\strut ##1}}}
\expandafter\def\csname PY@tok@kc\endcsname{\let\PY@bf=\textbf\def\PY@tc##1{\textcolor[rgb]{0.00,0.50,0.00}{##1}}}
\expandafter\def\csname PY@tok@kd\endcsname{\let\PY@bf=\textbf\def\PY@tc##1{\textcolor[rgb]{0.00,0.50,0.00}{##1}}}
\expandafter\def\csname PY@tok@kn\endcsname{\let\PY@bf=\textbf\def\PY@tc##1{\textcolor[rgb]{0.00,0.50,0.00}{##1}}}
\expandafter\def\csname PY@tok@kr\endcsname{\let\PY@bf=\textbf\def\PY@tc##1{\textcolor[rgb]{0.00,0.50,0.00}{##1}}}
\expandafter\def\csname PY@tok@bp\endcsname{\def\PY@tc##1{\textcolor[rgb]{0.00,0.50,0.00}{##1}}}
\expandafter\def\csname PY@tok@fm\endcsname{\def\PY@tc##1{\textcolor[rgb]{0.00,0.00,1.00}{##1}}}
\expandafter\def\csname PY@tok@vc\endcsname{\def\PY@tc##1{\textcolor[rgb]{0.10,0.09,0.49}{##1}}}
\expandafter\def\csname PY@tok@vg\endcsname{\def\PY@tc##1{\textcolor[rgb]{0.10,0.09,0.49}{##1}}}
\expandafter\def\csname PY@tok@vi\endcsname{\def\PY@tc##1{\textcolor[rgb]{0.10,0.09,0.49}{##1}}}
\expandafter\def\csname PY@tok@vm\endcsname{\def\PY@tc##1{\textcolor[rgb]{0.10,0.09,0.49}{##1}}}
\expandafter\def\csname PY@tok@sa\endcsname{\def\PY@tc##1{\textcolor[rgb]{0.73,0.13,0.13}{##1}}}
\expandafter\def\csname PY@tok@sb\endcsname{\def\PY@tc##1{\textcolor[rgb]{0.73,0.13,0.13}{##1}}}
\expandafter\def\csname PY@tok@sc\endcsname{\def\PY@tc##1{\textcolor[rgb]{0.73,0.13,0.13}{##1}}}
\expandafter\def\csname PY@tok@dl\endcsname{\def\PY@tc##1{\textcolor[rgb]{0.73,0.13,0.13}{##1}}}
\expandafter\def\csname PY@tok@s2\endcsname{\def\PY@tc##1{\textcolor[rgb]{0.73,0.13,0.13}{##1}}}
\expandafter\def\csname PY@tok@sh\endcsname{\def\PY@tc##1{\textcolor[rgb]{0.73,0.13,0.13}{##1}}}
\expandafter\def\csname PY@tok@s1\endcsname{\def\PY@tc##1{\textcolor[rgb]{0.73,0.13,0.13}{##1}}}
\expandafter\def\csname PY@tok@mb\endcsname{\def\PY@tc##1{\textcolor[rgb]{0.40,0.40,0.40}{##1}}}
\expandafter\def\csname PY@tok@mf\endcsname{\def\PY@tc##1{\textcolor[rgb]{0.40,0.40,0.40}{##1}}}
\expandafter\def\csname PY@tok@mh\endcsname{\def\PY@tc##1{\textcolor[rgb]{0.40,0.40,0.40}{##1}}}
\expandafter\def\csname PY@tok@mi\endcsname{\def\PY@tc##1{\textcolor[rgb]{0.40,0.40,0.40}{##1}}}
\expandafter\def\csname PY@tok@il\endcsname{\def\PY@tc##1{\textcolor[rgb]{0.40,0.40,0.40}{##1}}}
\expandafter\def\csname PY@tok@mo\endcsname{\def\PY@tc##1{\textcolor[rgb]{0.40,0.40,0.40}{##1}}}
\expandafter\def\csname PY@tok@ch\endcsname{\let\PY@it=\textit\def\PY@tc##1{\textcolor[rgb]{0.25,0.50,0.50}{##1}}}
\expandafter\def\csname PY@tok@cm\endcsname{\let\PY@it=\textit\def\PY@tc##1{\textcolor[rgb]{0.25,0.50,0.50}{##1}}}
\expandafter\def\csname PY@tok@cpf\endcsname{\let\PY@it=\textit\def\PY@tc##1{\textcolor[rgb]{0.25,0.50,0.50}{##1}}}
\expandafter\def\csname PY@tok@c1\endcsname{\let\PY@it=\textit\def\PY@tc##1{\textcolor[rgb]{0.25,0.50,0.50}{##1}}}
\expandafter\def\csname PY@tok@cs\endcsname{\let\PY@it=\textit\def\PY@tc##1{\textcolor[rgb]{0.25,0.50,0.50}{##1}}}

\def\PYZbs{\char`\\}
\def\PYZus{\char`\_}
\def\PYZob{\char`\{}
\def\PYZcb{\char`\}}
\def\PYZca{\char`\^}
\def\PYZam{\char`\&}
\def\PYZlt{\char`\<}
\def\PYZgt{\char`\>}
\def\PYZsh{\char`\#}
\def\PYZpc{\char`\%}
\def\PYZdl{\char`\$}
\def\PYZhy{\char`\-}
\def\PYZsq{\char`\'}
\def\PYZdq{\char`\"}
\def\PYZti{\char`\~}
% for compatibility with earlier versions
\def\PYZat{@}
\def\PYZlb{[}
\def\PYZrb{]}
\makeatother


    % Exact colors from NB
    \definecolor{incolor}{rgb}{0.0, 0.0, 0.5}
    \definecolor{outcolor}{rgb}{0.545, 0.0, 0.0}



    
    % Prevent overflowing lines due to hard-to-break entities
    \sloppy 
    % Setup hyperref package
    \hypersetup{
      breaklinks=true,  % so long urls are correctly broken across lines
      colorlinks=true,
      urlcolor=urlcolor,
      linkcolor=linkcolor,
      citecolor=citecolor,
      }
    % Slightly bigger margins than the latex defaults
    
    \geometry{verbose,tmargin=1in,bmargin=1in,lmargin=1in,rmargin=1in}
    
    

    \begin{document}
    
    
    \maketitle
    
    

    
    \begin{Verbatim}[commandchars=\\\{\}]
{\color{incolor}In [{\color{incolor}4}]:} \PY{c+c1}{\PYZsh{}\PYZsh{}\PYZsh{}\PYZsh{}\PYZsh{}\PYZsh{}\PYZsh{}\PYZsh{}\PYZsh{}\PYZsh{}\PYZsh{}\PYZsh{}\PYZsh{}\PYZsh{} Video1 \PYZsh{}\PYZsh{}\PYZsh{}\PYZsh{}\PYZsh{}\PYZsh{}\PYZsh{}\PYZsh{}\PYZsh{}\PYZsh{}\PYZsh{}\PYZsh{}\PYZsh{}}
        
        \PY{k}{class} \PY{n+nc}{Employee}\PY{p}{:}
            
            \PY{k}{def} \PY{n+nf}{\PYZus{}\PYZus{}init\PYZus{}\PYZus{}}\PY{p}{(}\PY{n+nb+bp}{self}\PY{p}{,} \PY{n}{first}\PY{p}{,} \PY{n}{last}\PY{p}{,} \PY{n}{pay}\PY{p}{)}\PY{p}{:}
                \PY{n+nb+bp}{self}\PY{o}{.}\PY{n}{first} \PY{o}{=} \PY{n}{first}
                \PY{n+nb+bp}{self}\PY{o}{.}\PY{n}{last} \PY{o}{=} \PY{n}{last}
                \PY{n+nb+bp}{self}\PY{o}{.}\PY{n}{pay} \PY{o}{=} \PY{n}{pay}
            
            \PY{k}{def} \PY{n+nf}{fullname}\PY{p}{(}\PY{n+nb+bp}{self}\PY{p}{)}\PY{p}{:}
                \PY{k}{return} \PY{n+nb+bp}{self}\PY{o}{.}\PY{n}{first}\PY{o}{+} \PY{l+s+s1}{\PYZsq{}}\PY{l+s+s1}{ }\PY{l+s+s1}{\PYZsq{}} \PY{o}{+}\PY{n+nb+bp}{self}\PY{o}{.}\PY{n}{last}
            
        
        \PY{n}{emp1} \PY{o}{=} \PY{n}{Employee}\PY{p}{(}\PY{l+s+s1}{\PYZsq{}}\PY{l+s+s1}{John}\PY{l+s+s1}{\PYZsq{}}\PY{p}{,}\PY{l+s+s1}{\PYZsq{}}\PY{l+s+s1}{Liver}\PY{l+s+s1}{\PYZsq{}}\PY{p}{,}\PY{l+m+mi}{60000}\PY{p}{)}
        \PY{n}{emp2} \PY{o}{=} \PY{n}{Employee}\PY{p}{(}\PY{l+s+s1}{\PYZsq{}}\PY{l+s+s1}{Mike}\PY{l+s+s1}{\PYZsq{}}\PY{p}{,}\PY{l+s+s1}{\PYZsq{}}\PY{l+s+s1}{Sweater}\PY{l+s+s1}{\PYZsq{}}\PY{p}{,}\PY{l+m+mi}{70000}\PY{p}{)}
        
        \PY{n+nb}{print}\PY{p}{(}\PY{n}{emp1}\PY{o}{.}\PY{n}{fullname}\PY{p}{(}\PY{p}{)}\PY{p}{)}
\end{Verbatim}


    \begin{Verbatim}[commandchars=\\\{\}]
John Liver

    \end{Verbatim}

    \begin{Verbatim}[commandchars=\\\{\}]
{\color{incolor}In [{\color{incolor}6}]:} \PY{c+c1}{\PYZsh{}\PYZsh{}\PYZsh{}\PYZsh{}\PYZsh{}\PYZsh{}\PYZsh{}\PYZsh{}\PYZsh{}\PYZsh{}\PYZsh{}\PYZsh{}\PYZsh{}\PYZsh{}\PYZsh{} Video 2 summmary \PYZsh{}\PYZsh{}\PYZsh{}\PYZsh{}\PYZsh{}\PYZsh{}\PYZsh{}\PYZsh{}\PYZsh{}\PYZsh{}\PYZsh{}\PYZsh{}\PYZsh{}\PYZsh{}\PYZsh{}\PYZsh{}\PYZsh{}\PYZsh{}\PYZsh{}\PYZsh{}}
        \PY{c+c1}{\PYZsh{}\PYZsh{}\PYZsh{} class variables}
        
        \PY{l+s+sd}{\PYZdq{}\PYZdq{}\PYZdq{}}
        \PY{l+s+sd}{Summary:}
        \PY{l+s+sd}{In this video, Corey taught as how to differentiate between a Class variable and instance variable, how they are related to each,}
        \PY{l+s+sd}{other, and when each of them is more useful over the other.}
        \PY{l+s+sd}{Class variables are variables that we set inside a class, and are shared among all instances. Corey gave a good example }
        \PY{l+s+sd}{where the number of total employs should be the same to every employ, no matter which employee we are referring to. Therefore,}
        
        \PY{l+s+sd}{emp\PYZus{}1.num\PYZus{}of\PYZus{}employ = emp\PYZus{}2.num\PYZus{}of\PYZus{}employ = Employee.num\PYZus{}of\PYZus{}employ}
        
        \PY{l+s+sd}{Instance variables are variables that are different from each instance. For example, the names and the pay for each employee. }
        \PY{l+s+sd}{They have to be different.}
        
        \PY{l+s+sd}{Corey also shows that class variables and instance variables are closely related, and that class variables are kind of }
        \PY{l+s+sd}{\PYZsq{}inherited\PYZsq{} to the \PYZsq{}self\PYZsq{} variables. To illustrate this, Corey shows an example of \PYZsq{}annual raise of pay\PYZsq{}. He initially creates }
        \PY{l+s+sd}{the class variable to show a case where annual raise is equal among all the employees. This variable of 1.04 was accessible }
        \PY{l+s+sd}{through each instance, and also through the class itself(obiviously). That is,}
        \PY{l+s+sd}{print(Employee.annual\PYZus{}raise)}
        \PY{l+s+sd}{print(emp\PYZus{}1.annual\PYZus{}raise)}
        \PY{l+s+sd}{print(emp\PYZus{}2.anual\PYZus{}rais)}
        \PY{l+s+sd}{all printed out 1.04.}
        
        
        \PY{l+s+sd}{However, using the .\PYZus{}dict\PYZus{}\PYZus{}  thing, Corey shows that the intances, emp\PYZus{}1 and emp\PYZus{}2 does not contain the annual\PYZus{}raise value. }
        \PY{l+s+sd}{Corey explains that if a variable is not found within an instance and programmers try to access the variable, }
        \PY{l+s+sd}{python automatically looks in in the variable of the instance\PYZsq{}s class, and then the more classes that the instance\PYZsq{}s class }
        \PY{l+s+sd}{inherits from.}
        
        
        \PY{l+s+sd}{Furthermore, if we access the class variable through an instance and then change it, python creates the variable within the }
        \PY{l+s+sd}{instance. We can check it by using the .\PYZus{}dict\PYZus{} thing. Corey shows that annual\PYZus{}raise key was created when he manually changed }
        \PY{l+s+sd}{the annual\PYZus{}raise value as 1.05 in the following way.}
        \PY{l+s+sd}{emp\PYZus{}1.annual\PYZus{}raise = 1.05}
        \PY{l+s+sd}{however, we know that the class variable remained the same at 1.04, when printing the class variable.}
        \PY{l+s+sd}{print(Employee.annual\PYZus{}raise)}
        
        
        \PY{l+s+sd}{==\PYZgt{} 1.04}
        
        \PY{l+s+sd}{\PYZdq{}\PYZdq{}\PYZdq{}}
\end{Verbatim}


    \begin{Verbatim}[commandchars=\\\{\}]
{\color{incolor}In [{\color{incolor}8}]:} \PY{c+c1}{\PYZsh{} class variable are the variables shared among all instances of a class}
        \PY{c+c1}{\PYZsh{} while instance variables can be unique like name, email and pay}
        \PY{c+c1}{\PYZsh{} class variables should be the same for each instance.}
        
        \PY{c+c1}{\PYZsh{} for the above Employee class we can use raise\PYZus{}amount as a class variable}
        
        \PY{c+c1}{\PYZsh{} before we create a class variable lets first hard code and see why a class varaible will be helpful}
        
        
        \PY{k}{class} \PY{n+nc}{Employee}\PY{p}{:}
            
            \PY{k}{def} \PY{n+nf}{\PYZus{}\PYZus{}init\PYZus{}\PYZus{}}\PY{p}{(}\PY{n+nb+bp}{self}\PY{p}{,} \PY{n}{first}\PY{p}{,} \PY{n}{last}\PY{p}{,} \PY{n}{pay}\PY{p}{)}\PY{p}{:}
                \PY{n+nb+bp}{self}\PY{o}{.}\PY{n}{first} \PY{o}{=} \PY{n}{first}
                \PY{n+nb+bp}{self}\PY{o}{.}\PY{n}{last} \PY{o}{=} \PY{n}{last}
                \PY{n+nb+bp}{self}\PY{o}{.}\PY{n}{pay} \PY{o}{=} \PY{n}{pay}
            
            \PY{k}{def} \PY{n+nf}{fullname}\PY{p}{(}\PY{n+nb+bp}{self}\PY{p}{)}\PY{p}{:}
                \PY{k}{return} \PY{n+nb+bp}{self}\PY{o}{.}\PY{n}{first}\PY{o}{+} \PY{l+s+s1}{\PYZsq{}}\PY{l+s+s1}{ }\PY{l+s+s1}{\PYZsq{}} \PY{o}{+}\PY{n+nb+bp}{self}\PY{o}{.}\PY{n}{last}
            
            \PY{c+c1}{\PYZsh{} create a  new method to give hike}
            
            \PY{k}{def} \PY{n+nf}{apply\PYZus{}raise}\PY{p}{(}\PY{n+nb+bp}{self}\PY{p}{)}\PY{p}{:}
                \PY{n+nb+bp}{self}\PY{o}{.}\PY{n}{pay} \PY{o}{=} \PY{n+nb}{int}\PY{p}{(}\PY{n+nb+bp}{self}\PY{o}{.}\PY{n}{pay} \PY{o}{*} \PY{l+m+mf}{1.04}\PY{p}{)}    \PY{c+c1}{\PYZsh{}hardcoding the raise amount to 4\PYZpc{}}
                
        \PY{n}{emp1} \PY{o}{=} \PY{n}{Employee}\PY{p}{(}\PY{l+s+s1}{\PYZsq{}}\PY{l+s+s1}{John}\PY{l+s+s1}{\PYZsq{}}\PY{p}{,}\PY{l+s+s1}{\PYZsq{}}\PY{l+s+s1}{Liver}\PY{l+s+s1}{\PYZsq{}}\PY{p}{,}\PY{l+m+mi}{60000}\PY{p}{)}
        \PY{n}{emp2} \PY{o}{=} \PY{n}{Employee}\PY{p}{(}\PY{l+s+s1}{\PYZsq{}}\PY{l+s+s1}{Mike}\PY{l+s+s1}{\PYZsq{}}\PY{p}{,}\PY{l+s+s1}{\PYZsq{}}\PY{l+s+s1}{Sweater}\PY{l+s+s1}{\PYZsq{}}\PY{p}{,}\PY{l+m+mi}{70000}\PY{p}{)}
        
        \PY{n+nb}{print}\PY{p}{(}\PY{n}{emp1}\PY{o}{.}\PY{n}{pay}\PY{p}{)}
        \PY{n}{emp1}\PY{o}{.}\PY{n}{apply\PYZus{}raise}\PY{p}{(}\PY{p}{)}   \PY{c+c1}{\PYZsh{} here this will apply the raise amount }
        \PY{n+nb}{print}\PY{p}{(}\PY{n}{emp1}\PY{o}{.}\PY{n}{pay}\PY{p}{)}
\end{Verbatim}


    \begin{Verbatim}[commandchars=\\\{\}]
60000
62400

    \end{Verbatim}

    \begin{Verbatim}[commandchars=\\\{\}]
{\color{incolor}In [{\color{incolor}11}]:} \PY{c+c1}{\PYZsh{} In the above code , we cannot modify the hike percentange like}
         \PY{c+c1}{\PYZsh{} emp1.raise\PYZus{}amount = somevalue}
         \PY{c+c1}{\PYZsh{} or Employee.raise\PYZus{}amount = somevalue}
         
         \PY{c+c1}{\PYZsh{} we currently doesnt have raise\PYZus{}amount attribute, we can create this as a class variable}
         
         
         \PY{k}{class} \PY{n+nc}{Employee}\PY{p}{:}
             
             \PY{c+c1}{\PYZsh{} class variable}
             \PY{n}{raise\PYZus{}amount} \PY{o}{=} \PY{l+m+mf}{1.04}
             
             \PY{k}{def} \PY{n+nf}{\PYZus{}\PYZus{}init\PYZus{}\PYZus{}}\PY{p}{(}\PY{n+nb+bp}{self}\PY{p}{,} \PY{n}{first}\PY{p}{,} \PY{n}{last}\PY{p}{,} \PY{n}{pay}\PY{p}{)}\PY{p}{:}
                 \PY{n+nb+bp}{self}\PY{o}{.}\PY{n}{first} \PY{o}{=} \PY{n}{first}
                 \PY{n+nb+bp}{self}\PY{o}{.}\PY{n}{last} \PY{o}{=} \PY{n}{last}
                 \PY{n+nb+bp}{self}\PY{o}{.}\PY{n}{pay} \PY{o}{=} \PY{n}{pay}
             
             \PY{c+c1}{\PYZsh{} create a  new method to give hike}
             
             \PY{k}{def} \PY{n+nf}{apply\PYZus{}raise}\PY{p}{(}\PY{n+nb+bp}{self}\PY{p}{)}\PY{p}{:}
                 \PY{c+c1}{\PYZsh{}self.pay = int(self.pay * 1.04)    \PYZsh{}hardcoding the raise amount to 4\PYZpc{}}
                 
                 \PY{c+c1}{\PYZsh{} here we can use the raise\PYZus{}amount as class instance or as an object instace}
                 
                 \PY{c+c1}{\PYZsh{}self.pay = int(self.pay * Employee.raise\PYZus{}amount) \PYZsh{} this is class instance}
                 \PY{n+nb+bp}{self}\PY{o}{.}\PY{n}{pay} \PY{o}{=} \PY{n+nb}{int}\PY{p}{(}\PY{n+nb+bp}{self}\PY{o}{.}\PY{n}{pay} \PY{o}{*} \PY{n+nb+bp}{self}\PY{o}{.}\PY{n}{raise\PYZus{}amount}\PY{p}{)}     \PY{c+c1}{\PYZsh{} this is object instance}
                 
                 \PY{c+c1}{\PYZsh{}self.pay = int(self.pay * raise\PYZus{}amount)   \PYZhy{} without self. or Employee it will error out}
                                                            \PY{c+c1}{\PYZsh{} saying raise\PYZus{}amount is not defined}
                 
                 \PY{c+c1}{\PYZsh{} when we use employee instance, the value will be same for all the objects}
                 \PY{c+c1}{\PYZsh{} and we cannot change the raise amount using emp1 or emp2 objects}
                 \PY{c+c1}{\PYZsh{} only way is to use \PYZdq{}Employee.raise\PYZus{}amount = new value\PYZdq{}}
                 
                 \PY{c+c1}{\PYZsh{} where  as when we use object instance(self.), we can change the value for }
                 \PY{c+c1}{\PYZsh{} which ever instance we would like to}
                 
         \PY{n}{emp1} \PY{o}{=} \PY{n}{Employee}\PY{p}{(}\PY{l+s+s1}{\PYZsq{}}\PY{l+s+s1}{John}\PY{l+s+s1}{\PYZsq{}}\PY{p}{,}\PY{l+s+s1}{\PYZsq{}}\PY{l+s+s1}{Liver}\PY{l+s+s1}{\PYZsq{}}\PY{p}{,}\PY{l+m+mi}{60000}\PY{p}{)}
         \PY{n}{emp2} \PY{o}{=} \PY{n}{Employee}\PY{p}{(}\PY{l+s+s1}{\PYZsq{}}\PY{l+s+s1}{Mike}\PY{l+s+s1}{\PYZsq{}}\PY{p}{,}\PY{l+s+s1}{\PYZsq{}}\PY{l+s+s1}{Sweater}\PY{l+s+s1}{\PYZsq{}}\PY{p}{,}\PY{l+m+mi}{70000}\PY{p}{)}
         
         \PY{n+nb}{print}\PY{p}{(}\PY{n}{emp1}\PY{o}{.}\PY{n}{pay}\PY{p}{)}
         \PY{n}{emp1}\PY{o}{.}\PY{n}{apply\PYZus{}raise}\PY{p}{(}\PY{p}{)}
         \PY{n+nb}{print}\PY{p}{(}\PY{n}{emp1}\PY{o}{.}\PY{n}{pay}\PY{p}{)}
\end{Verbatim}


    \begin{Verbatim}[commandchars=\\\{\}]
60000
62400

    \end{Verbatim}

    \begin{Verbatim}[commandchars=\\\{\}]
{\color{incolor}In [{\color{incolor}15}]:} \PY{n+nb}{print}\PY{p}{(}\PY{n}{Employee}\PY{o}{.}\PY{n}{raise\PYZus{}amount}\PY{p}{)}
         \PY{n+nb}{print}\PY{p}{(}\PY{n}{emp1}\PY{o}{.}\PY{n}{raise\PYZus{}amount}\PY{p}{)}
         \PY{n+nb}{print}\PY{p}{(}\PY{n}{emp2}\PY{o}{.}\PY{n}{raise\PYZus{}amount}\PY{p}{)}
         
         
         \PY{c+c1}{\PYZsh{} when we try to access an attribute on an instance , it will first check if that instance}
         \PY{c+c1}{\PYZsh{} contain that attribute and if it doesn\PYZsq{}t , it will check if the class or any other class }
         \PY{c+c1}{\PYZsh{} it inherits from contain that attribute}
         
         \PY{c+c1}{\PYZsh{} here when we access the raise\PYZus{}amount by using our instances, it won\PYZsq{}t have that attribute}
         \PY{c+c1}{\PYZsh{} so it gets it from the class}
\end{Verbatim}


    \begin{Verbatim}[commandchars=\\\{\}]
1.04
1.04
1.04

    \end{Verbatim}

    \begin{Verbatim}[commandchars=\\\{\}]
{\color{incolor}In [{\color{incolor}16}]:} \PY{c+c1}{\PYZsh{} in the above example, emp1 and emp2 namespaces will not have the raise\PYZus{}amount varaiable }
         \PY{c+c1}{\PYZsh{} to find out use \PYZus{}\PYZus{}dict\PYZus{}\PYZus{} method on the instance}
         
         \PY{n}{emp1}\PY{o}{.}\PY{n+nv+vm}{\PYZus{}\PYZus{}dict\PYZus{}\PYZus{}}
         
         \PY{c+c1}{\PYZsh{} in the below result that is no raise\PYZus{}amount attribute}
\end{Verbatim}


\begin{Verbatim}[commandchars=\\\{\}]
{\color{outcolor}Out[{\color{outcolor}16}]:} \{'first': 'John', 'last': 'Liver', 'pay': 62400\}
\end{Verbatim}
            
    \begin{Verbatim}[commandchars=\\\{\}]
{\color{incolor}In [{\color{incolor}17}]:} \PY{n}{Employee}\PY{o}{.}\PY{n+nv+vm}{\PYZus{}\PYZus{}dict\PYZus{}\PYZus{}}
         
         \PY{c+c1}{\PYZsh{} this will have the raise\PYZus{}amount attribute}
\end{Verbatim}


\begin{Verbatim}[commandchars=\\\{\}]
{\color{outcolor}Out[{\color{outcolor}17}]:} mappingproxy(\{'\_\_module\_\_': '\_\_main\_\_',
                       'raise\_amount': 1.04,
                       '\_\_init\_\_': <function \_\_main\_\_.Employee.\_\_init\_\_(self, first, last, pay)>,
                       'apply\_raise': <function \_\_main\_\_.Employee.apply\_raise(self)>,
                       '\_\_dict\_\_': <attribute '\_\_dict\_\_' of 'Employee' objects>,
                       '\_\_weakref\_\_': <attribute '\_\_weakref\_\_' of 'Employee' objects>,
                       '\_\_doc\_\_': None\})
\end{Verbatim}
            
    \begin{Verbatim}[commandchars=\\\{\}]
{\color{incolor}In [{\color{incolor}18}]:} \PY{c+c1}{\PYZsh{} when we change the raise\PYZus{}amount using the class instance , it will cascade the }
         \PY{c+c1}{\PYZsh{} new value to all the objects}
         
         \PY{n}{Employee}\PY{o}{.}\PY{n}{raise\PYZus{}amount} \PY{o}{=} \PY{l+m+mf}{1.05}
         
         \PY{n+nb}{print}\PY{p}{(}\PY{n}{Employee}\PY{o}{.}\PY{n}{raise\PYZus{}amount}\PY{p}{)}
         \PY{n+nb}{print}\PY{p}{(}\PY{n}{emp1}\PY{o}{.}\PY{n}{raise\PYZus{}amount}\PY{p}{)}
         \PY{n+nb}{print}\PY{p}{(}\PY{n}{emp2}\PY{o}{.}\PY{n}{raise\PYZus{}amount}\PY{p}{)}
\end{Verbatim}


    \begin{Verbatim}[commandchars=\\\{\}]
1.05
1.05
1.05

    \end{Verbatim}

    \begin{Verbatim}[commandchars=\\\{\}]
{\color{incolor}In [{\color{incolor}20}]:} \PY{c+c1}{\PYZsh{} if we were to set the raise amount of an instance rather than a class}
         
         \PY{n}{emp1}\PY{o}{.}\PY{n}{raise\PYZus{}amount} \PY{o}{=} \PY{l+m+mf}{1.07}
         
         \PY{n+nb}{print}\PY{p}{(}\PY{n}{Employee}\PY{o}{.}\PY{n}{raise\PYZus{}amount}\PY{p}{)}
         \PY{n+nb}{print}\PY{p}{(}\PY{n}{emp1}\PY{o}{.}\PY{n}{raise\PYZus{}amount}\PY{p}{)}
         \PY{n+nb}{print}\PY{p}{(}\PY{n}{emp2}\PY{o}{.}\PY{n}{raise\PYZus{}amount}\PY{p}{)}
         
         \PY{c+c1}{\PYZsh{} so here when we use emp1.raise\PYZus{}amount = value, it actually creates that attribute for this instance}
         \PY{c+c1}{\PYZsh{} we can see that using the .\PYZus{}\PYZus{}dict\PYZus{}\PYZus{} method }
\end{Verbatim}


    \begin{Verbatim}[commandchars=\\\{\}]
1.05
1.07
1.05

    \end{Verbatim}

    \begin{Verbatim}[commandchars=\\\{\}]
{\color{incolor}In [{\color{incolor}21}]:} \PY{n+nb}{print}\PY{p}{(}\PY{n}{emp1}\PY{o}{.}\PY{n+nv+vm}{\PYZus{}\PYZus{}dict\PYZus{}\PYZus{}}\PY{p}{)}
         
         \PY{c+c1}{\PYZsh{} now the raise\PYZus{}amount attribute will be available directly in the emp1 instance itself}
         \PY{c+c1}{\PYZsh{} so it won\PYZsq{}t use the class raise\PYZus{}amount attribute anymore}
         
         \PY{c+c1}{\PYZsh{} emp2 will not have this raise\PYZus{}amount and it will go and access class attribute}
\end{Verbatim}


    \begin{Verbatim}[commandchars=\\\{\}]
\{'first': 'John', 'last': 'Liver', 'pay': 62400, 'raise\_amount': 1.07\}

    \end{Verbatim}

    \begin{Verbatim}[commandchars=\\\{\}]
{\color{incolor}In [{\color{incolor}22}]:} \PY{c+c1}{\PYZsh{} if we were to add a new attribute to count the number of employees}
         \PY{c+c1}{\PYZsh{} it makes sense to have this attribute as a class attribute rather than an instace attribute}
         \PY{c+c1}{\PYZsh{} because we want to increment this attribute everytime a new object is created(or an employee object)}
         
         
         \PY{k}{class} \PY{n+nc}{Employee}\PY{p}{:}
             
             \PY{n}{no\PYZus{}of\PYZus{}emps} \PY{o}{=} \PY{l+m+mi}{0}
             \PY{n}{raise\PYZus{}amount} \PY{o}{=} \PY{l+m+mf}{1.04}
             
             \PY{k}{def} \PY{n+nf}{\PYZus{}\PYZus{}init\PYZus{}\PYZus{}}\PY{p}{(}\PY{n+nb+bp}{self}\PY{p}{,} \PY{n}{first}\PY{p}{,} \PY{n}{last}\PY{p}{,} \PY{n}{pay}\PY{p}{)}\PY{p}{:}
                 \PY{n+nb+bp}{self}\PY{o}{.}\PY{n}{first} \PY{o}{=} \PY{n}{first}
                 \PY{n+nb+bp}{self}\PY{o}{.}\PY{n}{last} \PY{o}{=} \PY{n}{last}
                 \PY{n+nb+bp}{self}\PY{o}{.}\PY{n}{pay} \PY{o}{=} \PY{n}{pay}
                 
                 \PY{c+c1}{\PYZsh{} incrementing the no of emps}
                 
                 \PY{n}{Employee}\PY{o}{.}\PY{n}{no\PYZus{}of\PYZus{}emps} \PY{o}{+}\PY{o}{=}\PY{l+m+mi}{1}    \PY{c+c1}{\PYZsh{} notice Employee. (its not self.)}
             
             \PY{k}{def} \PY{n+nf}{fullname}\PY{p}{(}\PY{n+nb+bp}{self}\PY{p}{)}\PY{p}{:}
                 \PY{k}{return} \PY{n+nb+bp}{self}\PY{o}{.}\PY{n}{first}\PY{o}{+} \PY{l+s+s1}{\PYZsq{}}\PY{l+s+s1}{ }\PY{l+s+s1}{\PYZsq{}} \PY{o}{+}\PY{n+nb+bp}{self}\PY{o}{.}\PY{n}{last}
             
             \PY{c+c1}{\PYZsh{} create a  new method to give hike}
             
             \PY{k}{def} \PY{n+nf}{apply\PYZus{}raise}\PY{p}{(}\PY{n+nb+bp}{self}\PY{p}{)}\PY{p}{:}
                 \PY{n+nb+bp}{self}\PY{o}{.}\PY{n}{pay} \PY{o}{=} \PY{n+nb}{int}\PY{p}{(}\PY{n+nb+bp}{self}\PY{o}{.}\PY{n}{pay} \PY{o}{*} \PY{n+nb+bp}{self}\PY{o}{.}\PY{n}{raise\PYZus{}amount}\PY{p}{)}
                 
         
         \PY{c+c1}{\PYZsh{} checking the no of employees before any objects}
         
         \PY{n+nb}{print}\PY{p}{(}\PY{n}{Employee}\PY{o}{.}\PY{n}{no\PYZus{}of\PYZus{}emps}\PY{p}{)}
                 
         \PY{n}{emp1} \PY{o}{=} \PY{n}{Employee}\PY{p}{(}\PY{l+s+s1}{\PYZsq{}}\PY{l+s+s1}{John}\PY{l+s+s1}{\PYZsq{}}\PY{p}{,}\PY{l+s+s1}{\PYZsq{}}\PY{l+s+s1}{Liver}\PY{l+s+s1}{\PYZsq{}}\PY{p}{,}\PY{l+m+mi}{60000}\PY{p}{)}
         \PY{n}{emp2} \PY{o}{=} \PY{n}{Employee}\PY{p}{(}\PY{l+s+s1}{\PYZsq{}}\PY{l+s+s1}{Mike}\PY{l+s+s1}{\PYZsq{}}\PY{p}{,}\PY{l+s+s1}{\PYZsq{}}\PY{l+s+s1}{Sweater}\PY{l+s+s1}{\PYZsq{}}\PY{p}{,}\PY{l+m+mi}{70000}\PY{p}{)}
         
         \PY{c+c1}{\PYZsh{} after employees creation}
         
         \PY{n+nb}{print}\PY{p}{(}\PY{n}{Employee}\PY{o}{.}\PY{n}{no\PYZus{}of\PYZus{}emps}\PY{p}{)}
\end{Verbatim}


    \begin{Verbatim}[commandchars=\\\{\}]
0
2

    \end{Verbatim}


    % Add a bibliography block to the postdoc
    
    
    
    \end{document}
